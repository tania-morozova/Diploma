\documentclass[11pt]{article}

\usepackage{a4wide}
\usepackage[utf8]{inputenc}
\usepackage[T2A]{fontenc}
\usepackage[russian]{babel}
\usepackage{graphicx}
\usepackage{color}
\usepackage[left=2.5cm,right=2.5cm,
    top=2.5cm,bottom=2.5cm,bindingoffset=0cm]{geometry}
\usepackage {amsthm}
\usepackage {amsmath}
\usepackage{amssymb}

\newtheorem{Def}{Определение}
\newtheorem{statement}{Утверждение}
\newtheorem{theorem}{Теорема}
\newenvironment{Proof}
{\par\noindent{\bf Доказательство.\\}} 
{\begin{flushright}$\Box$\end{flushright}}

\definecolor{gray}{rgb}{0.5,0.5,0.5}

\newcommand\Set[2]{\left\{ #1 \mid #2 \right\}}
\newcommand\Ref[1]{(\ref{#1})}
\newcommand\ftw[2]{\overline{#1,#2}}
\newcommand\RS{\Ref{system} }
\newcommand\beq{\begin{equation}}
\newcommand\eeq{\end{equation}}
\newcommand\dd[2]{\frac{\partial#1}{\partial#2}}


\begin{document}

\thispagestyle{empty}

\begin{center}
\ \vspace{-3cm}

\includegraphics[width=0.5\textwidth]{msu.eps}\\
{\scshape Московский государственный университет имени М.~В.~Ломоносова}\\
Факультет вычислительной математики и кибернетики\\
Кафедра системного анализа

\vfill

{\LARGE Выпускная квалификационная работа по теме}

\vspace{1cm}

{\Huge\bfseries <<Управление в моделях межвидового взаимодействия>>}
\end{center}

\vspace{1cm}

\begin{flushright}
  \large
  \textit{Студентка 415 группы}\\
  Т.~Е.~Морозова
  \textit{Научный руководитель}\\
  к.ф.-м.н., доцент И.~В.~Рублёв 

  \vspace{5mm}

 \end{flushright}

\vfill

\begin{center}
Москва, 2018
\end{center}

\newpage
\tableofcontents
\newpage

%=============Введение============
\section{Введение}

Рассматривается модель пищевой цепи с управлением без внутривидовой конкуренции, состоящая из четырех звеньев и описываемая следующей системой:

\beq
\left\{
\begin{aligned}
\label{system}
	\dot x_1 &= x_1(r_1 + u - b_1x_2), \\
	\dot x_2 &= x_2(-r_2 - b_2x_3 + c_2x_1), \\
	\dot x_3 &= x_3(-r_3 - b_3x_4 + c_3x_2), \\
	\dot x_4 &= x_4(-r_4 + c_4x_3).
\end{aligned}
\right.
\eeq

Здесь $x_i,\, i = \ftw{1}{4}$ --- численности популяций видов, $r_1 + u$ --- рождаемость первого вида,  $r_i,\, i  = \ftw{2}{4}$ --- смертности остальных видов, $b_i,\, i  = \ftw{1}{3}$ и $c_i,\, i  = \ftw{2}{4}$ отвечают за взаимодействие между популяциями. Все параметры строго положительны, а управление берется из интервала $U^* = [u_{min}, u_{max}].$

В данной работе основной целью является исследование свойств и синтеза управления для задачи быстродействия во множество положений равновесия.

%=============Общие свойства===========
\section{Общие свойства системы}

Из биологической интерпретации вытекает, что численность популяции не может быть отрицательной. В следующем утверждении будет показано, что система удовлетворяет этому свойству.

\begin{statement}
	Множество $\Set{x \in \mathbb{R}^4}{x_i > 0, i = \ftw{1}{4}}$ инвариантно относительно системы \RS.
\end{statement}
\begin{Proof}
	Из системы \RS видно, что при обнулении координат, обнуляются соответственно и выражения для $\dot x_i,$ так что координаты не могут поменять знак. Интегрируя в обратном времени, получим, что координаты не могут обнулиться.
\end{Proof}

Рассмотрим положения равновесия \RS как функцию управления: 
\begin{gather}
	P(u) = (P_1(u), P_2(u), P_3(u), P_4(u)),\; \text{где} \notag \\
	P_1(u) = \frac{r_2c_4 + b_2r_4}{c_2c_4}, \;\; P_2(u) = \frac{r_1 + u}{b_1}, \;\; P_3(u) = \frac{r_4}{c_4},\;\; P_4(u) = \frac{c_3(r_1+u) - r_3b_1}{b_1b_3}.
\end{gather}

Заметим, что от управления существенно зависят только вторая и четвертая координата, поэтому в дальнейшем будем обозначать $P(u) = (P_1, P_2(u), P_3, P_4(u)).$ \\

Множество положений равновесия $E = \Set{P(u)}{u \in U^*}$ представляет из себя отрезок прямой 
$$x_4 = \frac{c_3x_2 - r_3}{b_3}$$ 
в пространстве $\mathbb{R}^4,$ такой, что
$$x_2 \in \left[\frac{r_1 + u_{min}}{b_1}, \frac{r_1 + u_{max}}{b_1}\right], \; x_4 > 0$$ 
 и, вообще говоря, может быть пустым. Но в нашей работе мы предполагаем его непустоту, для этого введем следующее ограничение на параметры задачи:
$$u_{min} > \frac{r_3b_1 - c_3r_1}{c_3}.$$

Наша задача состоит в исследовании возможности перевода системы \RS во множество $E$ при кусочно-непрерывном управлении из $U^*.$ 

В работе \cite{MathBio} найден первый интеграл системы \RS:
\beq
	K(x,u) = x_1 - P_1\ln x_1 + \frac{b_1}{c_2}(x_2 - P_2(u)\ln x_2) + \frac{b_1b_2}{c_2c_3}(x_3 - P_3\ln x_3) + \frac{b_1b_2b_3}{c_2c_3c_4}(x_4 - P_4(u)\ln x_4).
\eeq

Докажем, что функция $K(x,u)$ сильно выпукла по $x$ и имеет минимум по $x$ в точке $(P(u),u).$

\begin{statement}
	Функция $K(x,u)$ сильно выпукла на любом выпуклом ограниченном подмножестве $\mathbb{R}_+^4,$ а ее глобальный минимум по $x$ достигается в точке $(P(u),u).$
\end{statement}
\begin{Proof}
	Рассмотрим гессиан функции $K(x,u):$
	$$H = \text{diag} \left(\frac{P_1}{x_1^2}, \; \frac{P_2(u)b_1}{c_2x_2^2}, \; \frac{P_3b_1b_2}{c_2c_3x_3^2}, \; \frac{P_4(u)}{c_2c_3c_4x_4^2}\right).$$
	Очевидно, он больше нуля при всех $x_i > 0,$ следовательно, функция выпукла.\\
	Обозначим 
	$$K_1(x_1) = x_1 - P_1\ln x_1, \; K_2(x_2,u) = \frac{b_1}{c_2}(x_2 - P_2(u)\ln x_2),$$
	$$K_3(x_3) = \frac{b_1b_2}{c_2c_3}(x_3 - P_3\ln x_3), \; K_4(x_4,u) = \frac{b_1b_2b_3}{c_2c_3c_4}(x_4 - P_4(u)\ln x_4,$$ тогда $K(x,u) =  K_1(x_1) + K_2(x_2,u) + K_3(x_3) + K_4(x_4,u).$  Поскольку 
	$$K_1'(x_1) = 1 - \frac{P_1}{x_1}, \; K_2'(x_2,u) = \frac{b_1}{c_2} - \frac{P_2(u)b_1}{c_2x_2},$$
	$$K_3'(x_3) = \frac{b_1b_2}{c_2c_3} - \frac{b_1b_2P_3}{c_2c_3x_3}, \; K_4(x_4,u) = \frac{b_1b_2b_3}{c_2c_3c_4} - \frac{b_1b_2b_3P_4(u)}{c_2c_3c_4x_4},$$ 
	глобальный минимум $K_i$ достигается при $x_i = P_i,$ следовательно, глобальный минимум $K(x,u)$ достигается в точке $(P(u),u).$ \\
	 Докажем сильную выпуклость. Возьмем $x_i \leqslant \mu, i = \ftw{1}{4}, \text{где } \mu \geqslant \max\Set{P_i}{i = \ftw{1}{4}}.$ Тогда $H \geqslant \delta\cdot I,$ где  
	 $$\delta = \min\left(P_1, \frac{P_2(u)b_1}{c_2}, \frac{P_3b_1b_2}{c_2c_3}, \frac{P_4(u)b_1b_2b_3}{c_2c_3c_4}\right)/\mu^2.$$
	 Таким образом $\langle Hx,x \rangle \geqslant \langle \delta Ix,x \rangle = \delta \|x\|^2,$ что и завершает доказательство.
\end{Proof}

%Введем функцию $\tilde K(x,u) = K(x,u) - K(P(u),u).$ Очевидно, она также сильно выпукла, неотрицательна и равна нулю в точке минимума $(P(u),u).$ Таким образом, наша задача сводится к поиску минимума функции $\tilde K(x,u)$ по $u$ из $U^*.$\\
%\textcolor{blue}{Вообще пока нет особого смысла в $\tilde K(x,u).$ Или есть?.. Тут надо доказать как-то, что если $u^*(t) \in \bar U^* : K(x,u^*) \leqslant K(x,u)\; \forall x, \forall u \in U^*, \text {то } \exists t > t_0 : x(t,t_0,x_0 \mid u*) \in E, \;\; \bar U^* = \Set{u(t)}{u(t) \in U^* \forall t}.$}
%
%Таким образом, будем искать $\min\limits_{u(t) \in \bar U^*} K(x,u).$ 
%Посчитаем производную:
%$$\frac{dK(x,u)}{du} = -(P_2)'_u\frac{b_1}{c_2}\ln x_2 - (P_4)'_u\frac{b_1b_2b_3}{c_2c_3c_4}\ln x_4 = -\frac{c_4\ln x_2 + b_2\ln x_4}{c_2c_4}.$$
%Видим, что минимум достигается на кривой 
%\beq
%	x_2 = x_4^{-\frac{b_2}{c_4}}.
%\eeq
%\textcolor{blue}{Видимо, это кривая переключения управления???.\\}
%
%Проверим, есть ли на этой кривой неподвижные точки:
%$$
%	\left\{
%	\begin{aligned}
%		x_4 &= x_2^{-\frac{c_4}{b_2}}, \\
%		x_4 &= \frac{c_3x_2 - r_3}{b_3}
%	\end{aligned}
%	\right.
%$$
%Так как все параметры задачи положительны, то эти кривые имеют ровно одну точку пересечения в области $x_2 > 0, x_4 > 0.$

%Будем искать $\min\limits_{u \in U^*} K(x,u).$ 
%он каким-то хуем приведет нас в положение равновесия. 
\textcolor{gray}{Будем максимально уменьшать $K(x,u)$, то есть гасить <<энергию системы>>, что будет приближать нас  положению равновесия.\\}

Посчитаем производную $\frac{dK(x,u_0)}{dt}$ при некотором $u_0$ в силу системы \RS:

$$\frac{dK(x,u_0)}{dt} = \dd{K(x,u_0)}{x_1}\dot x_1 + \dd{K(x,u_0)}{x_2}\dot x_2 + \dd{K(x,u_0)}{x_3}\dot x_3 + \dd{K(x,u_0)}{x_4}\dot x_4 = (u - u_0)\left(x_1 - \frac{b_2r_4 + c_4r_2}{c_2c_4}\right).$$

Заметим, что равновесие по $x_1$ разделяет все пространство на две области. В области $x_1 > \frac{b_2r_4 + c_4r_2}{c_2c_4}$ производная будет минимальной при $u = u_{min}, \, u_0 = u_{max},$ а при $x_1 < \frac{b_2r_4 + c_4r_2}{c_2c_4}$ --- наоборот, то есть при $u = u_{max}, u_0 = u_{min}.$ При $x_1 = \frac{b_2r_4 + c_4r_2}{c_2c_4}$ возникает \textcolor{gray}{<<особый режим>>.}

Выясним, может ли \textcolor{gray}{<<особый режим>>} наступать на временном отрезке ненулевой меры.

...

%Особые режимы
Выпишем функцию Гамильтона-Понтрягина для нашей системы и проверим, могут ли быть особые режимы.
$$H(\psi, x, u) = \psi_1x_1(r_1 + u - b_1x_2) + \psi_2x_2(-r_2 - b_2x_3 + c_2x_1) + \psi_3x_3(-r_3 - b_3x_4 + c_3x_2) + \psi_4x_4(-r_4 + c_4x_4),$$

где  $\psi)t) \in C, \; \psi(t) \ne 0 \; \text{и удовлетворяет сопряженной системе системе:}$

$$
\left\{
\begin{aligned}
	\dot \psi_1 &= -(\psi_1(r_1 + u - b_2x_2) + \psi_2x_2c_2), \\
	\dot \psi_2 &= -(-b_1\psi_1x_1 + \psi_2(-r_2 - b_2x_3 + c_2x_1) + \psi_3x_3c_3), \\
	\dot \psi_3 &= -(-b_2\psi_1x_2 + \psi_3(-r_3 - b_3x_4 + c_3x_2) + \psi_4x_4c_4), \\
	\dot \psi_4 &= -(b_3\psi_3x_3 + \psi_4(-r_4 + c_4x_3)).
\end{aligned}
\right.$$

Из принципа максимума $H(\psi(t), x(t), u^*(t)) = \sup\limits_{u \in U^*} H(\psi,x,u).$

Посчитаем производную $H$ по $u:$

$$H'_u = \psi_1x_1, \text{ откуда } u^* = \begin{cases} u_{max}, & \psi_1x_1 \geqslant 0, \\  u_{min}, & \psi_1x_1 < 0.\end{cases}$$

Так как в нашей системе $x_1(t) > 0,$ особый режим будет возникать, если $\psi_1(t) = 0$ на ненулевом промежутке времени. Рассмотрим последовательно $\dot \psi_i$ и убедимся, что все сопряженные переменные нулевые.

Если $\psi_1(t) = 0$ на промежутке $[t_1, t_2],$ то $\dot \psi_1 = 0$ на этом же временном отрезке. Но тогда из первого сопряженного уравнения $\psi_2 = 0,$ следовательно и $\dot \psi_2 = 0, \;  t \in [t_1, t_2],$ а тогда из второго сопряженного уравнения $\psi_3 = 0, \;  t \in [t_1, t_2].$ Аналогично $\dot \psi_3 = 0, \;  t \in [t_1, t_2]$ и из третьего сопряженного уравнения $\psi_4 = 0, \;  t \in [t_1, t_2],$ что противоречит невырожденности $\psi(t).$

Таким образом, особых режимов в системе не возникает и управление кусочно-постоянное.

%
%Пусть это так. Тогда на этом отрезке $\dot x_1 = 0,$ значит, $x_2 = P_2(u),$ тогда, если мы положим управление постоянным, то $\dot x_2 = 0,$ тогда $x_3 = P_3, \, \dot x_3 = 0$ и $x_4 = P_4(u).$ Значит, мы попали в положение равновесия. 
%
%Заметим, что $\forall \delta > 0, \text{если} |x_1 - P_1| \geqslant \delta, \text{то} \frac{dK(x,u_0)}{dt} \leqslant -\delta (u - u_0) = -\delta |u_{max} - u_{min}|.$


%===========Библиография============
\clearpage
\newpage
\begin{thebibliography}{0}
\bibitem{MathBio} {\it Massarelli~N., Hoffman~K., Previte~J.~P.} Effect of parity on productivity and sustainability of Lotka–Volterra food chains. // Mathematical Biology. December of 2014. 1609--1626.
\end{thebibliography}










\end{document}